\problem{Induction 3}

Prove the following?
Let $x \neq 1$ be a real number. For all $n \in \mathcal{N}$, 

$\frac{(x^n - 1)}{x - 1} = (x^(n - 1) + x^(n - 2) + .. + x + 1$

\solution

\part 

\begin{theorem}
    The formula $\frac{(x^n - 1)}{x - 1}$ describes the summation $(x^(n - 1) + x^(n - 2) + .. + x + 1$ for all $n \in \mathcal{N}$.
\end{theorem}

\begin{proof}
    Proceeding by induction, it is first shown that the formula describes the summation for $n = 1$, i.e. $\frac{x^n - 1}{x - 1} = \frac{x - 1}{x - 1} = 1$.
    Then, the induction hypothesis is assumed to be true, such that $\frac{x^n - 1}{x - 1} = (x^(n - 1) + x^(n - 2) + .. + x + 1)$.
    So, $\frac{x^{n + 1} - 1}{x - 1} = (x^n + x^(n - 1) + x^(n - 2) + .. + x + 1)$.
    Which simplifies to $\frac{x^{n + 1} - 1}{x - 1} = x^n + \frac{x^n - 1}{x - 1} = \frac{x^n - 1 + (x^n)(x - 1)}{x - 1} = \frac{x^{n + 1} - 1}{x - 1}$, which is what was to be shown.
    By induction, the formula $\frac{(x^n - 1)}{x - 1}$ describes the summation $(x^(n - 1) + x^(n - 2) + .. + x + 1$ for all $n \in \mathcal{N}$.
\end{proof}
