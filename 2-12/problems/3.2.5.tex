\problem{Induction 2}

Let $m$ and $n \in \mathcal{N}$. Define what it means to say that $m$ divides $n$. Now prove that for all $n \in \mathcal{N}$, 6 divides $n^3 - n$.

\solution

\part 

It is said that $m$ divides $n$ if there exists a natural number $o$ such that $om = n$.
\begin{theorem}
    For all $n \in \mathcal{N}$, 6 divides $n^3 - n$.
\end{theorem}

\begin{proof}
    Proceeding by induction, it is first shown that 6 divides $n = 2$ (as $n = 1$ is a trivial, but true, case, where $o = 0$).
    $n^3 - n = 6$, which is obviously divided by $6$ with $o = 1$.
    Assume the induction hypothesis, namely that $n^3 - n = 6 * o$, where $o$ is a natural number.
    For $n + 1$, the expression becomes $(n + 1)^3 - (n + 1) = 6 * p$, where $p$ is another natural number.
    This expands to $n^3 + 3n^2 + 2n = 6 * p$. 
    Using the induction hypothesis, $n^3 - n = 6 * o$, $(6 * o) + 3n^2 + 3n = 6 * p$
    Then, $3(n)(n + 1) = 6(p - o)$
    Then, $(n)(n + 1) = n^2 + n = 2(p - o)$, which simply means that $n^2 + n$ must be even for any $n$ in the natural numbers. 
    If $n$ is even ($n = 2a$), then $n^2 + n = 4a^2 + 2a = 2(2a^2 + a)$, and the expression is even.
    If $n$ is odd ($n = 2b + 1$), then $n^2 + n = 4b^2 + 4b + 1 + 2b + 1 = 2(2b^2 + 3b + 1)$, and the expression is even.
    Therefore, by induction, for all $n \in \mathcal{N}$, 6 divides $n^3 - n$.
\end{proof}

