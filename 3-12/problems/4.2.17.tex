\problem{4.2.17}

Show by giving an example that immediate successors and immediate predecessors are not necessarily unique.

\solution

\part 

Non-unique immediate successors and immediate predecessors are possible if a set has a partial ordering, but not a total one.
For instance, consider an ordering over $\{a, b, c\}$ where $a < b$ and $a < c$, but no other orderings are defined. 
In this case, the successors to $a$, which are $b$ and $c$, are clearly not unique. 
If the relation is inverted, then the same is true for predecessors.
