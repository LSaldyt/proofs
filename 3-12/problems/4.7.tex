\problem{4.7}

Is it possible for a partially ordered set to have both a least element and a minimal element that is \emph{not} a least element?

\solution

\part 

\begin{theorem}
It is not possible for a partially ordered set to have both a least element and a minimal element that is \emph{not} a least element.
\end{theorem}

\begin{proof}
Proceeding by contradiction, consider a poset $A, \leq$ which has a least element, $a$. 
Least simply means that for all elements $x \in A$, $a \leq x$.
So, suppose there there existed a minimal element, call it $b$, which was \emph{not} a least element.
The definition of minimal is weaker, and simply states that $b$ is not greater than (i.e. $\leq$) any element in $A$.
This scenario is not possible, by cases:
\begin{enumerate}
    \item Either the minimal element $b$ is not related to the least element, $a$, and thus $a$ is not a least element.
    \item Or, $b$ is equal to $a$, in which case $b$ is a least element.
\end{enumerate}
Since both cases lead to a contradiction, the theorem holds.
\end{proof}
