\problem{4.3.23}

Show that the following relations $r$ on the specified set $S$ are equivalence relations. In each case do this in two ways:
\begin{itemize}
\item By identifying the equivalence classes and noting that they partition $S$.
\item By showing directly that $r$ is reflexive, symmetric, and transitive.
\end{itemize}

\solution

\part 

\begin{enumerate}
    \item $S = \{p : p \text{ is a person in Ohio}\}$. A rel B represents that person A and person B were born in the same year. \\
        Let $\Omega_r$ be the set of equivalence classes, a collection of subsets of $S$ associated with relation $r$, where each $T_\alpha$ in $\Omega_r$ is the set of relatives of $\alpha$. 
        Years are natural numbers with a lower bound. Thus, there is a finite number of them. 
        In each year, if person A is born in the same year as person B, then the reverse is also true (Thus $r$ is symmetric).
        Thus, each year has a set of people born in that year, and this set is an equivalence class $T_\alpha$.
        Each $T_\alpha$ is disjoint from other $T_\beta$s, and covers all people because a person must be born in a particular year, and only one particular year.
        This shows that the equivalence classes partition $S$.
        Now, $r$ is also reflexive and transitive because each person is born in the same year as themself, and if person $a$ is born in the same year as person $b$ and person $b$ is born in the same year as person $c$, then the people $a, b$ and $c$ (of interest: $a$ and $c$) are all born in the same year.
        So, $r$ is symmetric, reflexive, and transitive.
    \item $S = \mathcal{Z}$. a rel b represents that $\abs{a} = \abs{b}$ \\
        It is only true that $\abs{a} = \abs{b}$ when either $a = b$ or $a = -b$ (or $-a = b$, though $a$ and $b$ are general). So, to start with, this is actually a combination of two equivalence relations.
        Consider the equivalence classes $T_\alpha$. For any element $a$ in $S$, it is known that $T_\alpha$ includes two elements, and no others: $a$ and $-a$.
        $r$ is symmetric because the underlying relation $=$ is symmetric.
        Because $r$ is symmetric, $T_\alpha$ is disjoint from other equivalence classes, but in total $\Omega_r$ includes all elements in $S$.
        This shows that the equivalence classes partition $S$.
        Now, $r$ is also reflexive and transitive because each number is equal to itself, regardless of sign, and the transitive property holds on the underlying equality relationship.
\end{enumerate}

