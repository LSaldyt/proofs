\problem{4.2.15}

Let $A$ be a partially ordered set. Prove that if $A$ has a greatest element, then the greatest element is unique. (Assume two greatest elements and show they are the same).

\solution

\part 

\begin{theorem}
If $A$ has a greatest element, then the greatest element is unique.
\end{theorem}

\begin{proof}
Suppose $A$ has a greatest element, $a$. For all $y$ in $a$, $y \sim x$. Suppose that there were another greatest element, $b$, which satisfied the same property. This implies that $a \sim b$ and $b \sim a$. Then, because any partial ordering satisfies anti-symmetry by definition, $a = b$, and thus the greatest element of $A$ is unique.
\end{proof}
