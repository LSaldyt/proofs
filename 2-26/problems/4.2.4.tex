\problem{4.2.4}

Let $A$ be a set. Show that $\mathcal{P}(A)$ need not be totally ordered under the relation $\subseteq$.

\solution

\part 

\begin{theorem}
$\mathcal{P}(A)$ is not totally ordered under the relation $\subseteq$.
\end{theorem}

\begin{proof}
Recall that a set $A$ is said to be totally ordered if it has a relation which is anti-symmetric, transitive, and satisfies the "connex property": $a ~ b$ or $b ~ a$ for any $a, b$ in the set $A$. While $\subseteq$ satisfies anti-symmetry and transitivity for $\mathcal{P}(A)$, it does not satisfy the connex property. For instance, the powerset of $\{0, 1\}$ contains the elements $\{0\}$ and $\{1\}$. Let these be the variables $a$ and $b$ in the connex property, and it is clear that it is
not satisfied (because $\{0\} \not\subseteq \{1\}$ and $\{1\} \not\subseteq \{0\}$).
\end{proof}
