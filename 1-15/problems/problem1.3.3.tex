\problem{Quantification and Ordering (1.3.3)}

Consider the predicate about integers ``x = 2y``, which contains two free variables. 
There are six distinct ways to use quantification to turn this predicate into a statement (Why six?).
Find all six statements and determine the truth or falsehood of each.

\solution

\part 

Given ``there exists`` and ``for all``, there are six distinct quantifications because identical quantifiers cannot be swapped to make a new quantification, but otherwise order matters when quantifying. Assume x and y are taken from the real numbers.

\begin{enumerate}
    \item There exists an $x$ and there exists a $y$ such that $x = 2y$. This is true. It is saying that given some arbitrary number, its double exists.
    \item There exists an $x$ such that for all $y$, $y$ $x = 2y$. This is false. It is saying that given some arbitrary number, all of the real numbers are it doubled.
    \item There exists a $y$ such that for all $x$, $y$ $x = 2y$. This is false for the same reason as 2, except halved instead of doubled.
    \item For all $y$ and for all $x$, $x = 2y$. This is false. It is saying that all real numbers are doubles/halves of each other.
    \item For all $y$, there exists an $x$ such that $x = 2y$. This is true, since $x$ is assigned after $y$.
    \item For all $x$, there exists a $y$ such that $x = 2y$. This is true, since $y$ is assigned after $x$.
\end{enumerate}
